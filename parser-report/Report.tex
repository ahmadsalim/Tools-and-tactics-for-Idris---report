\documentclass[a4paper]{article}%{llncs}
\DeclareTextCommandDefault{\nobreakspace}{\leavevmode\nobreak\ }
\usepackage{listings} % Nice code-boxes
\usepackage{nameref}
\usepackage{graphicx} % For \includegraphics
\usepackage{float}    % For in-line images
\usepackage{amsmath}
\usepackage{amsfonts}
\usepackage{url} %For urls
\usepackage[hidelinks]{hyperref}
\usepackage{caption} % For advanced captioning
\usepackage{subcaption}
\usepackage{authblk}
\usepackage{algorithm}
\usepackage[noend]{algpseudocode}
\usepackage{fontspec}
\usepackage{fancybox}

\newfontfamily\unicodemonofamily{Menlo}
\graphicspath{{Images/}}
\bibliographystyle{splncs}

\makeatletter
\newenvironment{CenteredBox}{\begin{Sbox}}{\end{Sbox}\centerline{\parbox{\wd\@Sbox}{\TheSbox}}}% And output it centered
\makeatother

% make a proper TOC despite llncs, %http://code.google.com/p/decidr/wiki/LatexLLNCSTableOfContents
\setcounter{tocdepth}{3}
\setcounter{secnumdepth}{5}
\makeatletter
% end TOC fix
% Pseudocode fix
\newcommand*\Let[2]{\State #1 $\gets$ #2}
\algrenewcommand\alglinenumber[1]{
  {\sf\footnotesize#1}}
% end Pseudocode fix
\newtheorem{exm}{Example} %Example theorem
\newtheorem{dfnt}{Definition} %Definition theorem
\begin{document}

\title{Parsing Idris \\ \normalsize{A part of the Tools and Tactics for Idris project}}
\author{Ahmad Salim Al-Sibahi (\texttt{asal@itu.dk}) \\IT University of Copenhagen\\Supervisors: \\David Raymond Christiansen (\texttt{drc@itu.dk})\\ Dr. Peter Sestoft (\texttt{sestoft@itu.dk})}
\date{\today}


\maketitle
\lstset{basicstyle=\scriptsize\unicodemonofamily, captionpos=b, extendedchars=false, numbers=left, stepnumber=3, firstnumber=1, language=Haskell}

Idris is a dependently-typed programming language with support for tactics-based theorem proving. This report outlines my work on
understanding what dependently typed languages are, how it is possible to do proves using tactics and finally implement a tactic that
allows proofs by induction myself.
The induction tactic is based on eliminators, which represent a logic-based approach to computation using an inductive data-type with some similarities to pattern matching
in the programming world.

The work resulted in an algorithm for generating eliminators which supported all various kinds of inductive data-structures, except inductive-recursive ones. An induction
tactic based on the eliminators was created successfully, albeit with minor limitations that should be easily solvable in the future.

The following pull requests signifies my work on eliminators and the induction tactic: \url{https://github.com/idris-lang/Idris-dev/pull/673} and \url{https://github.com/idris-lang/Idris-dev/pull/713}.


%\tableofcontents
%\clearpage
\section{Introduction}
\label{sec:Introduction}

Idris\cite{brady2013idris} is a strict pure dependently-typed programming language inspired by Haskell\cite{marlow2010haskell} and ML-style languages\cite{milner1997definition}.
As a dependently-typed language, Idris allows types to be indexed by ordinary values such as the natural numbers
in addition to being parametrised with type variables.
This makes Idris a powerful language that allows one to reason more formally about one's programs and makes it suitable for a variety of applications
such as writing verified programs or embedded domain specific languages (EDSLs) where many errors can be reported at compile time.

Yet all this flexibility comes with a price.
The grammar is vastly complicated since there is no distinction between types and terms, and the focus on EDSLs has brought in more features such as syntactic extensions.
Partly because of this and partly because of the white space sensitive Haskell-like syntax; Idris is not a particularly trivial language to parse.

As such one carefully need to evaluate the way the grammar is written, and write the parser in such way that there is little room for ambiguity and that the user can get practical error messages.
Furthermore the source code must be tracked carefully to provide sensible errors all the way through the system, and that external tools can use this to provide additional features such as clickable source code.

A parser already exists for Idris, written in the Parsec\cite{leijen2001parsec} library. Yet, because of all the advanced features of Idris and the way the parser is written, error location is highly inaccurate, the error messages are marginally useful
and it can not be used by other tools for analysis or completion due to lack of incremental parsing.

In this paper I will outline my work regarding restructuring of the Idris parser, such that the grammar is more sensible, the parser provides better errors and that the source code is more accurately tracked.
The paper will be structured as follows. In Section~\ref{sec:IdrisSyntax} I will discuss the Idris grammar and what particular difficulties there can be when trying to parse it.
Section~\ref{sec:Trifecta} discusses the particular parser technology used, and what advantages and challenge it involves. Section~\ref{sec:Solution} describes the specific implementation of the parser, and which improvements was done.
I evaluate the solution in Section~\ref{sec:Evaluation} and discuss future work in Section~\ref{sec:FutureWork}. Lastly I conclude in Section~\ref{sec:Conclusion}.

\section{Idris Syntax}
\label{sec:IdrisSyntax}

The syntax of Idris is heavily inspired by the syntax of Haskell with some modifications to better accommodate the dependent types-based programming paradigm and EDSLs.

In this section I will describe by example the interesting differences and highlight some of the features that makes creating a parser non-trivial.

Figure~\ref{fig:simpleidrisprog} showcases a simple inductive data declaration for natural numbers, and a pattern-matching definition for a function `add' which computes the sum of two naturals.
The only noticeable difference in syntax from an ordinary Haskell program is that `:' is used for type declarations instead of `::'.

\begin{figure}[H]
\begin{CenteredBox}
\lstinputlisting[firstline=1, lastline=7]{examples/grammar.idr}
\end{CenteredBox}
\caption{Simple Idris program}
\label{fig:simpleidrisprog}
\end{figure}

Similar to Haskell, Idris also allows one to define custom operators with user-defined fixity and precedence; but because of the lack of distiction between types and terms in Idris these
operators can now appear anywhere in the program, even at type level. This can be seen in Figure~\ref{fig:opsandlambdas} where the data type for existential types is using the operator `***' as name.

In Figure~\ref{fig:opsandlambdas} a type for length-indexed list, `vectors', is also defined. One can notice that this definition, similarly to the declaration for existential types, uses an extended style for data declarations
that is more suitable for specifying dependencies in types. Some of the arguments are put between curly braces which specify that the given argument is implicit, although this is only allowed syntactically at top-level declarations.

Finally in the figure, a function `take\_while' is defined.
A thing one can notice that both the `***' operator and a lambda declaration were allowed at the type level.
Another thing one should notice is the indentation-sensitive where the `with' and `where'-clauses must conform to a specific indentation to be syntactically valid.

\begin{figure}[H]
\begin{CenteredBox}
  \lstinputlisting[firstline=9, lastline=24]{examples/grammar.idr}
\end{CenteredBox}
\caption{Operators and lambdas}
\label{fig:opsandlambdas}
\end{figure}

For convenience and improved readability of EDSLs, Idris allows further expansion of the syntax of the language. In Figure~\ref{fig:syntaxextensions} there has been defined two syntactic extensions,
one for the `take\_while' function and one for the existential data type. The syntax definitions (line 1-2) can define new keywords or syntactic markers (specified as either simple identifiers or quoted in a string),
capture other expressions which are delimited in square brackets (e.g. `[vect]' in the example), or capture names which are delimited in curly braces (e.g. `\{x\}') that can be used for binding.
This makes syntactic extensions quite powerful, and the only thing one cannot seemingly extend the language with is recursive grammar definitions.

At lines 10-11 in Figure~\ref{fig:syntaxextensions} there exist an expression which uses these syntactic extensions as any other built-in syntax.
Rather than being baked into Idris, many things like `if'-expressions are in practice defined using syntax extensions in the library,

\begin{figure}[H]
\begin{CenteredBox}
  \lstinputlisting[firstline=27, lastline=37]{examples/grammar.idr}
\end{CenteredBox}
\caption{Syntax extensions}
\label{fig:syntaxextensions}
\end{figure}


\section{Trifecta}
\label{sec:Trifecta}
Trifecta\cite{kmett2013trifecta} is a monadic parser combinator library created by Edward Kmett with focus on providing good tools for error reporting and incremental parsing.

In this section I will explain what parser combinators are, and how Trifecta as a library is designed such that the combinators provide some desirable features.

\subsection{Parser Combinators}
\label{sub:ParserCombinators}
Traditionally there are two ways of creating a parser: either by hand-writing the parser, or by giving the grammar as input to a parser generator such as yacc\cite{johnson1975yacc} and generating one.

Parser combinator libraries\cite[Chapter~16]{osullivan2008realworldhaskell}\cite{hutton1996monadic} provide a way to help creating hand-written parsers by providing, often in the form of a EDSL, a set of generic higher-order functions called parser combinators.

Parser combinators produce new parsers by taking other parsers as input, and then can supply features as alternation, sequencing, choice, repetition and cursor movement.
The constituting parsers can be either parser created by other parser combinators, or basic parsers which accepts a character from a limited range.

\subsubsection{Combinators by example}
\label{ssub:Combinators by example}

\begin{figure}[H]
\begin{CenteredBox}
  \lstinputlisting[firstline=8]{examples/parse.hs}
\end{CenteredBox}
\caption{Parsing a login}
\label{fig:loginparse}
\end{figure}

Figure~\ref{fig:loginparse} shows a simple example of a parser for parsing login information given in the following format ``id--password'', where id is either a phone number or a self-chosen user-name
and password is a string of alphanumerical characters.

The choice for id is created by using the `<|>' combinator, which tries the first alternative and if it fails without consuming any input will try the second alternative.
Since there is overlap between the definition of phone number and username, phone number must be wrapped in a try-combinator which ensures that input can be restored by backtracking such that no input is consumed and the
other alternative can be tried.

Because the parser is monadic, sequence multiple parsers is conveniently available using ordinary monadic bind (or do-notation as done in the example), and thus it was easy to provide a way of parsing the following separator and password after
parsing the id.

In the example there are three repetition parser combinators used: `count' which specifies an exact number of repetitions, `many' which specifies zero or more repetitions and `some' which specifies one or more repetitions;
although many more are usually available in such libraries.

Finally, one can observe that the phone number parser uses an combinator `option' which allows specification of an optional part of the input and the combinator `$\ast$>' which ignores the result of the left operand and returns
the result of the right operand.

\subsection{Features of Trifecta}
\label{sub:FeaturesofTrifecta}
While there are many parser combinator libraries for Haskell, Trifecta provides the following desirable features:
\begin{description}
  \item[Incremental parsing] When working with external tools, the input might change quite often.
                             It is therefore desirable if only the relevant part of the input is parsed again, and Trifecta achieves this by using monoidal parsing\cite{kmett2009iteratorsparsecandmonoids}.
                             This feature can allow things like semantic highlighting and code completion to be more easily implemented.
  \item[Error reporting] Trifecta has a focus on nice diagnostics and priniting. If a parser fails, it doesn't just show what it expected but also nicely formats and points to the place of error. Additionally
                         Trifecta provides ways to highlight semantic errors in the code using the same type of spanning it uses for syntactic errors.
\end{description}

\section{Solution}
\label{sec:Solution}
In many ways transitioning the parser from Parsec to Trifecta and rewriting the grammar is non-trivial.
In this section I will outline my effort on improving the parser and highlighting the interesting parts, namely formalization of the grammar and improvement on the error reporting and locating.
This entails that in the interest of readability, some of the more mechanical and structural parts will not be explained in detail although a lot of effort was put into that part.

\subsection{Formalising the Grammar}
\label{sub:FormalisingtheGrammar}
Part of the challenge of writing a new parser for Idris is that at the time there were no formalization of grammar in Idris.
This entailed two issues:
\begin{enumerate}
  \item There was no official reference regarding syntax for users of Idris. This meant that it was hard to find what the correct syntax is if there was an error and the only way to correct this is either:
    \begin{itemize}
      \item by trying to isolate the bug in a structured fashion and try to guess how it should be corrected
      \item or by trying to use a lot of time analysing the parser code and see what the legal syntax should have been
    \end{itemize}
\item There were some grammatical inconsistencies, such that some programs were falsely accepted (e.g.\ programs that were missing a `\}') and some didn't behave as expected thereby being falsely rejected (e.g.\ proof-blocks didn't supported verbose but not indentation-style syntax, unlike do-blocks which supported both).
\end{enumerate}

The formalization to an official BNF grammar was further complicated by the fact that Idris supports an indentation based syntax, and furthermore it supports user-defined syntax extensions. As such the resulting grammar is incomplete due to fact that
it is impossible to formalize all possible syntax extensions and is not totally sound because it does not reflect all of the lexical properties of indentation.

The latter could be improved by specifying the grammar in an extended BNF syntax like the one used
by Adams\cite{adams2013principled}, but at the time of analyses there were too many inconsistencies regarding indentation in order to completely reflect the actual syntax and thus this was omitted.

The actual translation process was done by carefully analysing the existing parser structure, noting any seeming inconsistencies and correcting them in the grammar.
The final result is available in Appendix~\ref{sec:IdrisFormalizedGrammar}.

\subsection{Documenting for Failures}
\label{sub:DocumentingforFailures}
%Column tracking, parser documentation
Another issue there was with the existing parser was that grammar rules were improperly documented so instead of error messages showing what legal types of grammar it expected, it instead showed a list of tokens that can appear somewhat random
for the intended user. For example if a user had mistyped something in the start of an expression, it would show all operators, keywords and marker symbols that were possible at that point instead of just showing that
an expression was expected.

Therefore some of the work done on the parser was to provide a human-readable string explanation for each grammar rule using the `<?>' combinator. Figure~\ref{fig:docparrule} shows an example of such annotation.
Additionally comments and white space was documented in such way that the parser didn't report them as expected, unlike in the previous version.

\begin{figure}[H]
\begin{CenteredBox}
\lstinputlisting{examples/errordoc.hs}
\end{CenteredBox}
\caption{Documenting Parser Rules}
\label{fig:docparrule}
\end{figure}


\subsection{The Parser with Fear of Commitment}
\label{sub:TheParserwithFearofCommitment}
One of the most challenging parts of the grammar re-factoring was that many of the rules were written in such way that they didn't commit to any of the branches in a list of alternatives.

This meant that each time there was an error in one of the alternatives, the parser had to backtrack all the way back to the starting point.

In addition to being exponentially slow and memory requiring, in the case that
none of the branches succeeding it will show a location at the start of the grammar rule instead of the location where the possible error happens. In the case of top-level declarations, this can be multiple lines and columns away from the actual location,
and finding the correct location is extremely hard for the programmer.

Figure~\ref{fig:noncomittingparser} shows one of the rules, were one can observe that all alternatives are wrapped in the try-combinator meaning it will never commit to any alternative in case of an error.

\begin{figure}[H]
\begin{CenteredBox}
  \lstinputlisting[firstline=1, lastline=7]{examples/commitment.hs}
\end{CenteredBox}
\caption{Non-committing Parser Rule}
\label{fig:noncomittingparser}
\end{figure}

To fix this usually two things are required:
\begin{enumerate}
  \item Reordering of rules such that less ambiguous rules comes before more ambiguous rules
  \item Minimizing the span of the try-combinator such that only the ambiguous part of covered, i.e. until a keyword or marking character
\end{enumerate}
Figure~\ref{fig:comittingparser} shows the committing version of the rule shown in Figure~\ref{fig:noncomittingparser}, where some of the alternatives has been reordered to avoid backtracking.
It should be noted that this was a simple example of such re-factoring, and more complex rules such as the one for data declaration required significantly more effort.

\begin{figure}[H]
\begin{CenteredBox}
  \lstinputlisting[firstline=9, lastline=15]{examples/commitment.hs}
\end{CenteredBox}
\caption{Committing Parser Rule}
\label{fig:comittingparser}
\end{figure}

\subsection{Improving Source Tracking}
\label{sub:ImprovingSourceTracking}
In the old parser there was tracking of source code in the abstract syntax tree, but some of the tracking was highly inaccurate and was missing column information.

In some elements like atomic identifiers or references, the location was first retrieved after the token was parsed; This meant that if there was a lot of white space the reference would have been at the end
of the white space, instead of at the start of the token.

For these tokens, a correct was made so that the location was retrieved \textit{before} a token was parsed.
This additionally had the benefit of allowing easier access to highlighting relevant identifiers by simply selecting all text from the stored location to white space.

Furthermore, all the abstract syntax trees at various level were altered to provide column information such that multiple items on the same line were distinguishable.
This also highlighted an old bug where the structural equality of the abstract syntax tree was dependent on the location, and thus two widely different variables could have been assumed to be the same if they were on the same line.

\section{Evaluation}
\label{sec:Evaluation}
In this section I will highlight some of the improvements that were discovered and fixed in the new parser, and what challenges there still are because of the way the grammar is constructed.
Additionally I will highlight the improvements in error handling and the reception by the community both \underline{non-empirically} and in terms of reported bugs.

\subsection{Syntactic Corrections and Challenges}
\label{sub:SyntacticCorrectionsandChallenges}
The following bugs in the syntax was corrected:
\begin{enumerate}
  \item Proof and tactic blocks allow indented-style syntax, similarly to do blocks
  \item Documentation comments can be nested inside existing comments
  \item Where is optional on instances
  \item Tactic sequences can hold more than two tactics
  \item Using and parameter blocks can be empty
  \item Checks that parentheses are matched
\end{enumerate}

The only challenging item in the current implementation of the parser is the parsing of nested parenthesized expressions, which currently is exponentially slow.
This is due to ambiguity in grammar where operator slices e.g. `($\backslash\backslash$ x)' are hard to distinguish from lambdas `($\backslash$ y => y * x)' and thus all parenthesized expression require full lookahead.

This way nested parenthesized expression can do exponential backtracking.

\subsection{Error Report Improvements}
\label{sub:ErrorReportImprovements}
As a showcase for error reporting improvements I will highlight a couple of examples where a simple error resulted some bad locations and suggestions by the old parser,
but shows a clear placement and presentation of error in the new parser.

\begin{figure}[H]
\begin{minipage}{\linewidth}
\begin{CenteredBox}
\lstinputlisting{examples/error-report.idr}
\end{CenteredBox}
\subcaption{File with error} \par \medskip \vfill
\begin{CenteredBox}
\lstinputlisting[language={}]{examples/error-report.old.out}
\end{CenteredBox}
\subcaption{Old parser report} \par \medskip \vfill
\begin{CenteredBox}
\lstinputlisting[language={}]{examples/error-report.new.out}
\end{CenteredBox}
\subcaption{New parser report} \par \medskip \vfill
\end{minipage}
\caption{Improvements in Error Reporting}
\label{fig:improvementsinerrorrepoting}
\end{figure}

Figure~\ref{fig:improvementsinerrorrepoting} shows an Idris program where the user had accidentally used `::' for constructor type signatures similar to Haskell
instead of `:' used in Idris. As can be observed in the figure, the old parser reports that the error is at line 3 (where type is declared, not the constructor) and
suggests that it expected one token from a long list of unrelated ones.

The new parser substantially improves the error location by pointing directly at the place of error and showing a part of the code that was affected. Furthermore
it only suggests that the type signature should be fixed, which is the correct error in this case.

\begin{figure}[H]
\begin{minipage}{\linewidth}
\begin{CenteredBox}
\lstinputlisting{examples/error-report2.idr}
\end{CenteredBox}
\subcaption{File with error} \par \medskip \vfill
\begin{CenteredBox}
\lstinputlisting[language={}]{examples/error-report2.old.out}
\end{CenteredBox}
\subcaption{Old parser report} \par \medskip \vfill
\begin{CenteredBox}
\lstinputlisting[language={}]{examples/error-report2.new.out}
\end{CenteredBox}
\subcaption{New parser report} \par \medskip \vfill
\end{minipage}
\caption{Improvements in Error Reporting, cont.}
\label{fig:improvementsinerrorrepotingc}
\end{figure}

Figure~\ref{fig:improvementsinerrorrepotingc} shows similarly that the old parser error report were basically unusable. The user had accidentally
forgot to finish the expression of an infix operator and instead of complaining that it had missed the rest of the operation it instead complains at the start of the function
clause saying that it didn't expect the argument given.

As can be seen, the new parser correctly highlights that the parser didn't expect end-of-file but an expression and the user can immediately correct this bug instead
of hunting down the error at the completely wrong place.

\subsection{Community Reception}
\label{sub:CommunityReception}
The parser described in this paper has officially been included in the Idris language, and is now the solely used parser.

The general reception has been positive by the community and in the word of the language designer, it has been described as following in the changelog: ``New parser implementation with more precise errors.''

Initially there has been less than 10 bugs reported, many of which already existed in the previous parser, but were undiscovered to bad error messages.
However, the number of bugs reported regarded parsing has decreased substantially and the last reported bug was reported more than a month ago since writing this report.

\textbf{NOTE:} Add results for HTML clickable code thing if there is time


\section{Future Work}
\label{sec:FutureWork}
While there has been lot of effort improving the parser, not all of the advantages of the parser technology was utilized.
In the future work, I would suggest the following improvements that could be made:

\begin{description}
  \item[Spanning the source code] Currently the source code is only tracked by a cursor in the abstract syntax tree, and while this allows good semantic error reporting and OK tooling support,
        a better solution would entail tracking the full span of a structure in the AST. This would allow easier access to pretty printing the affected code, and
        using the parser technology to better print and highlight the errors.
  \item[Utilising incremental parsing] Trifecta supports incremental parsing, and it could provide many useful features in regards to tooling. As part of future work
                                       it could be of great interest to utilize this feature to allow for things like semantic highlighting and code completion.
  \item[Refactoring semantics out of the parser] Currently the parser does a lot of semantic checking inside the parser structure, in order to ensure
                                                 that some things are correct by construction.

                                                 For example implicit parameters are not allowed in function arguments
                                                 and the parser currently disallows writing implicit argument syntax in the type signatures where function arguments
                                                 are expected. This results in a parser error saying that it didn't expect the implicit syntax, and while this is correct
                                                 it might be confusing for a new user. As such, many of these semantic checks should be moved out of the parser and
                                                 refactored in a separate syntax checking step which provide better error report for such situations.
\end{description}

\section{Conclusion}
\label{sec:Conclusion}
In this project I have successfully formalised the grammar and rewritten the parser for the Idris language.
I did this by analysing the existing parser, fixing any inconsistencies and rewritten the new parser in such way the error reporting was improved and locations were tracked better.

I also conclude that the resulting work has been received well by the community, and that it can be seen as a great success that the resulting work now forms the basis for
the Idris language syntax and parsing.

Finally, while there were improvements to be made, changing the parser technology to Trifecta has provided us with a great start for improving tooling and semantic error reporting.


%\let\Section\section  %add "References" to TOC
%\def\section*#1{\Section{#1}}

\bibliography{Report}

\appendix
\section{Idris Formalized Grammar}
\label{sec:IdrisFormalizedGrammar}
  \lstinputlisting[language={}, basicstyle=\tiny\ttfamily]{examples/grammar.txt}


\end{document}
